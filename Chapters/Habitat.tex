\chapter{Habitat of ice reservoirs}

For centuries, in the Himalayan mountain ranges, local cultures have believed that glaciers are alive. And
what’s more, that certain glaciers can have different genders including male and female. These people ‘breed’
new glaciers by grafting together—or marrying—fragments of ice from male and female glaciers, then covering them
with charcoal, wheat husks, cloths, or willow branches so they can reproduce in privacy. These glacierets
transform into fully active glaciers that grow each year with additional snowfall. Those then serve as lasting
reserves of water that farmers can use to irrigate their crops. Over the years, these practices have inspired
other cultures, where people are creating their own artificial ice reservoirs (AIRs) and applying them to solve
serious modern challenges around water supplies.

\section{Natural vs Artificial Ice Reservoirs}
