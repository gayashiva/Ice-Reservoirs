\chapter{Habitat of ice reservoirs}

AIRs cannot be built anywhere. They require a water source sufficiently above and weather conditions cold enough
to amass a seasonal stock of ice. This imposes several meteorological and topographical  requirements for the
chosen construction location. The meteorological requirements can be used to identify favourable regions
worldwide whereas the topographical requirements can be used to pinpoint the construction site within the
respective region. Below we detail these requirements and propose methodologies for finding construction sites
satisfying them. 

\section{Requirements for AIR construction}

\subsection{Meteorological requirements}

AIRs prefer colder, drier and less-cloudy regions. Our results quantify this preference based on maximum ice
volume estimations of the Swiss and the Indian AIRs (see Paper I). The Swiss AIRs have little utility as a water
reservoir due to their small size and short survival duration. Therefore, construction of AIRs should not be
carried out in locations less favourable than the Swiss site. 


\subsection{Topographical requirements}

The water source of an AIR could be either a spring, stream or lake. Springs are the ideal water source since
they are easy to transport via pipelines to the construction site due to their relatively warm temperatures.
Other water sources tend to freeze within the pipeline during tranport. It is pointless to use lakes as water
sources unless there is utility in draining and converting them to ice structures. This is the case for glacial
lakes which can be drained to prevent flash floods while using the water supply to harvest ice structures.

AIRs prefer shadowed valleys. This is because their melt rate is driven by solar radiation (see Paper I).

\section{Quantifying AIR construction potential of any locations}

However, it is challenging to determine a methodology to compare meteorological conditions of two locations. Our
suggested methodology is described in

% \section{Natural vs Artificial Ice Reservoirs}
