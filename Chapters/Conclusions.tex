\chapter{Conclusions}

Chapter 5 provides conclusions based on research findings from data collected on AIRs in Switzerland and India,
as well as discussion and recommendations for future research. This Chapter will review the purpose of the
study, research questions, literature review, and findings of the study. It will then present conclusions,
discussion of the conclusions, and recommendations for practice and for further research.

\section{Summary}

Cryosphere fed irrigation networks are completely dependant on the timely availability of meltwater from
glaciers, snow and permafrost. With the accelerated decline of glaciers, these irrigation networks have failed
to deliver adequate water to sustain agricultural output and take advantage of the complete growing season. As a
consequence, many mountain villages have either been abandoned or lie on the brink of desertification.

In the past few decades, artificial ice reservoir (AIR) technologies have provided much needed relief to these
water-stressed communities. These strategies revolve around augmenting their glacial ice reservoirs with
man-made ones that provide supplementary irrigation during the spring. 

Summary of the literature review

From folklore to science, from art to technology, these practices of reclaiming glacial water have come a long
way. 

Empirical data and studies focussing on AIRs are sparse. Depending on the relative influence of weather
conditions and fountain characteristics, AIRs typically show high variability in their ice volume dynamics.
Because small-scale processes, complex feedbacks and non-linearities govern their evolution, accessing the
response of AIRs to the location and fountain chosen is challenging and only feasible if backed up with
comprehensive field data. 

Summary of the methodology

In the context of the observed present and predicted global glacier shrinkage, the development of such water
storage technologies is crucial to ensure continued survival of mountain communities.

Summary of the findings

\section{Conclusions}

The main objective of this thesis was thus to improve our understanding about the response of AIRs to changes in
their construction location and fountain characteristics. With a special focus on icestupas, here defined as
vertical AIRs, volume changes were investigated in detail based on the comparison of weather patterns, fountain
characteristics and volume observations between several AIRs. A mass and energy model was created which allowed
ice volume estimation of icestupas. AIR radius, area and volume were recorded using DEMs produced from drone
flights. The fountain characteristics were calibrated from the observed radius and discharge rates. The model
parameters were calibrated from some volume observations. The rest of this DEM dataset were used to validate the
modelled volume evolution. The ice volume dynamics due to the difference in weather patterns of the Swiss
location and Indian location can be summarised as follows: 

% \begin{itemize} 

% \item[\tiny{$\blacksquare$}] Colder temperatures and lower humidity led to higher sublimation and lower
%   conduction.
%  
% \item[\tiny{$\blacksquare$}] Cloudy days increase shortwave radiation impact.
%  
% \end{itemize}

\begin{itemize} 

\item[\tiny{$\blacksquare$}] Icestupa's volume evolution depends on meteorological conditions and fountain
  characteristics.

\item[\tiny{$\blacksquare$}] Icestupas gain volume due to higher vapour losses and not inspite of it. 

\item[\tiny{$\blacksquare$}] Icestupa's shape decreases direct shortwave radiation impact.

\item[\tiny{$\blacksquare$}] Icestupas gain volume due to higher vapour losses and not inspite of it. 

\item[\tiny{$\blacksquare$}] Icestupas suffer high water losses.
 
\end{itemize}

\begin{itemize} 

\item[\tiny{$\blacksquare$}] Colder, drier and less cloudy construction locations form long-lasting AIRs with
  higher maximum ice volumes. 

\item[\tiny{$\blacksquare$}] Fountains that produce smaller droplets form larger AIRs with higher slope. 

\end{itemize}


\section{Discussion}

\section{Recommendations}

\section{Suggestions for future research}

\section{Final thoughts}
