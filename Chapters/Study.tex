\chapter{Study of ice reservoirs}


\section{Data Collection}

We chose two villages in the Swiss Alps and the Indian Himalayas called Guttannen and Gangles to collect the
required datasets described above. These two locations exhibit drastically different weather patterns (see
Table) owing to their latitude, longitude and altitude differences. This enabled us to highlight the
meteorological influences on ice volume evolution (RQ 1).

The Guttannen site (46.66 $\degree$N, 8.29 $\degree$E) is situated in the Berne region, Switzerland and has an
altitude of 1047 $m$ a.s.l. In the winter (Oct-Apr), mean daily minimum and maximum air temperatures vary
between -13 and 15 $\degree C$. Clear skies are rare, averaging around 7 days during winter. Daily winter
precipitation can sometimes be as high as 100 $mm$. These values are based on 30 years of hourly historical
weather data measurements \citep{meteoblueClimateGuttannen2021}. Several AIRs were constructed by the Guttannen
Bewegt Association, the University of Fribourg and the Lucerne University of Applied Sciences and Arts during
the winters of 2020-22.

The Gangles site (34.22 $\degree$N, 77.61 $\degree$E) is located around 20 km north of Leh city in the Ladakh
region, lying at 4025 $m$ a.s.l.. The mean annual temperature is $5.6 \, \degree C$, and the thermal range is
characterized by high seasonal variation. During January, the coldest month, the mean temperature drops to $-7.2
\, \degree C$. During August, the warmest month, the mean temperature rises to $17.5 \, \degree C$
\citep{Nusser_2012}. Because of the rain shadow effect of the Himalayan Range, the mean annual precipitation in
Leh totals less than 100 $mm$, and there is high interannual variability. Whereas the average summer rainfall
between July and September reaches 37.5 $mm$, the average winter precipitation between January and March amounts
to 27.3 $ mm$ and falls almost entirely as snow. AIRs were constructed here as part of the Ice Stupa Competition
by the Himalayan Institute of Alternatives, Ladakh (HIAL). 


