\chapter{Worship of ice reservoirs}

For centuries, in the Himalayan mountain ranges, local cultures have believed that glaciers are alive. And
what’s more, that certain glaciers can have different genders including male and female. These people ‘breed’
new glaciers by grafting together—or marrying—fragments of ice from male and female glaciers, then covering them
with charcoal, wheat husks, cloths, or willow branches so they can reproduce in privacy. These glacierets
transform into fully active glaciers that grow each year with additional snowfall. Those then serve as lasting
reserves of water that farmers can use to irrigate their crops. Over the years, these practices have inspired
other cultures, where people are creating their own artificial ice reservoirs (AIRs) and applying them to solve
serious modern challenges around water supplies.

\section{An old history}

It is not known who started the practice of glacier grafting, but according to oral history it is as old as the
local communities. According to legend, when the people of Baltistan learnt of the Mongol army advancing towards
them from the north in the early 13th century, they came up with an ingenious way to stop them. As the inhabited
valleys were only accessible through narrow passes, they decided to block the entry way by building a glacier.
This successfully prevented the Mongol invasion and, crucially, it also solved the locals’ other big problem:
water scarcity.


\section{Glacier Marriages}

The people of Gligit Baltistan believe that glaciers are living entities. That’s why a combination of female and
male ice was absolutely necessary. The male glacier – called ‘po gang’ locally – gives off little water and
moves slowly, while a ‘female glacier’ – or ‘mo gang’ – is a growing glacier that gives off a lot of water.

The glaciers that people help to grow are the fruit of the sacred union between a mother glacier and a father
glacier. They get married and have offspring. The selection of an appropriate site for this marriage is of
utmost importance, and a suitable spot must fulfil a list of conditions. It should be located at an altitude of
at least 4000 or 5000 metres above sea level; it should be on a gentle slope, where it should have minimal
exposure to sunlight, thus a north-facing mountain side is preferable. For most of the expert glacier grafters,
the presence of permafrost or ice on the site is another key requirement. 

Once a suitable spot is selected, the expedition can be planned. The bride and the groom – the female and the
male glacier, preferably from different villages – are chosen and the marriage can be planned. The glacier
grafting usually takes place in November, when the local temperatures oscillate around zero. A 12-man party
carries the pieces of female ice in woven baskets, another 12 men carry the male ice, the water drawn from the
Indus river is carried traditionally in 12 gourd bottles, but sometimes clay pots or goatskins are also
required, as well as charcoal and wheat husks or sawdust, which act as insulators for the ice. The last
ingredient is salt, which, according to some glacier grafters, helps protect the new glacier from impurities.
The bride and groom party walk from different sites and meet in a certain spot to climb together to the glacier
growing site, but no greetings are exchanged, as the people involved in the ceremony must remain silent until
the ice is deposited in its new home. They walk continuously without having a break, but if the distance is too
much and rest is required, they do not put their loads on the ground, instead hanging the baskets on trees, or
on walking sticks if nothing else is available. Each man has to carry around 15 to 25 kilograms of ice, walking
in cold air, silently up the mountains, for a day or more. Once they reach the glacier growing site, they
deposit their valuable loads. The ice lumps and water bottles are placed in between the boulders, or in a small
cave, or sometimes in a specially dug pit, and covered with layers of salt, charcoal and sawdust. The silence is
broken as religious leaders recite verses of the Quran and say prayers for the success of the glacier marriage
and for protection from the djinns. Once the male and female glaciers are placed in their new home and covered,
a man from the party of glacier grafters stands up and offers his life for the success of the process. His
symbolic sacrifice is matched by the actual sacrifice of a goat – its meat is distributed to a charity, because
prayers are more likely to be answered if accompanied by an act of charity. They will not visit the place for at
least three years, so as not to disturb the glacier. It is said that a person who disturbs the glacier before
its maturation will die. The celebrations continue in the village with traditional songs and prayers, alongside
festive food and the joy of the accomplished mission.

\section{From folklore to science}

Myths, legends and superstitions are ways of knowing. But they need to be translated to the language of science.

According to Ingvar Tveiten, a researcher from Norway, the account of the glacier development process presented
by a glacier grafter from Balghar bears a strong resemblance to the definition of the formation of rock
glaciers. According to a description by a Balghar local: “First the ice slips down into the rocks where it grows
roots. Then it starts to break the rocks bringing them up. Then the glacier comes forward. This has happened
where they did the glacier growing.” Tveiten, who conducted field research in Baltistan, concludes that “glacier
growing is typically performed […] in a terrain that is conducive to the accumulation of snow by avalanching and
snow slips. The presence of permafrost at these locations is likely to contribute to ice accumulating […] Thus,
glacier growing is conducted at locations which are already very prone to ice accumulation, and may explain why
glacier growing is perceived to work.” Here it is, traditional knowledge translated into the language of
science.

This suggests that the technique was developed as a result of the local
people’s deep understanding of local environmental processes.

When it comes to past projects, there is only anecdotal evidence available.
