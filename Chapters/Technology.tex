\chapter{Technology of ice reservoirs}

Adequate crop production assumes a central role in the lives of mountain farmers, making the availability of
irrigation water indispensable. The chief components of that resource are meltwater from glaciers and seasonal
snow cover. However, seasonal melting of glaciers occurs quite late because of the regions's high altitude,
which delays the availability of glacial meltwater until June. This seasonal water scarcity makes it essential
to provide supplementary irrigation in order to take advantage of the complete growing season. 

The construction of horizontal AIRs on south-facing slopes is regarded not only as a way to cope with water
scarcity but also as an adaptation to climate change \citep{norphelArtificialGlacierHigh2009}. These structures
are built as a cascading series of rock walls in the river beds to reduce runoff velocity and guide meltwater
into shadowed areas (see Fig.). The resulting shallow pools begin to freeze as temperatures drop in winter, and
ice accumulates. These AIRs are constructed as close to the villages as possible. Being at a lower altitude than
the natural glaciers (see Fig. ), AIRS begin melting in April, providing irrigation just in time for the start
of the agricultural season (see Fig. ). Chewang Norphel, a well known engineer of the Leh Nutrition Project,
introduced this innovation of local technology to Ladakh in the 1980s and 1990s \citep{vinceGlacierMan2009}.

A spirit of improvisation guides the design and construction of AIRs making it difficult to make a quantitative
comparison from site to site.

A typical Icestupa just requires a fountain nozzle mounted on a supply pipeline. The water source is usually a
spring or a glacial stream. Due to the altitude difference between the pipeline input and fountain output, water
ejects from the fountain nozzle as droplets that eventually lose their energy and accumulate as ice.  The
fountain is manually activated during the winter nights and is raised, through addition of metal pipes, when
significant ice accumulates below.

\section{The invention of AIR technology}

\begin{figure}[t]
\includegraphics[width=12cm]{Figures/AIR_forms.jpg}

\caption{(a) Schematic overview of the position of artificial ice reservoirs. These constructions are located at
  altitudes between the glaciers and the irrigation networks in the cultivated areas. (b) Horizontal ice
  reservoirs at 3900 m, located above the village of Nang, Ladakh. The cascade is composed of a series of loose
  masonry walls ranging in height from 2 to 3 $m$, which help freeze water for storage. (c) Vertical ice
reservoirs at 3600 m, located above the village of Phyang, Ladakh. They are made using fountain systems. Adapted
from: \cite{nusserLocalKnowledgeGlobal2016}}

\label{fig:AIRforms}
\end{figure}

\section{The evolution of construction strategies}

THis effort to create AIRs can be related to traditional water harvesting technologies like the zing, which are
small tanks where meltwater is collected through the use of an expensive and intricate network of channels. The
mountain oases of the Hindu Kush and Karakoram ranges have similar irrigation networks
\citep{nusserLocalKnowledgeGlobal2016}.

AIRs are a natural evolution of Ladakh's agricultural system.

AIRs are engineered systems exploiting gravity and freezing winter temperatures to amass a seasonal stock of
ice.

\subsection{Building Horizontal AIRs}

\subsection{Building Vertical AIRs}
