\chapter{Heritage of ice reservoirs}

This chapter provides conclusions based on research findings from data collected on AIRs in Switzerland and India,
as well as discussion and recommendations for future research. This Chapter will review the purpose of the
study, research questions, literature review, and findings of the study. It will then present conclusions,
discussion of the conclusions, and recommendations for practice and for further research.

\section{Summary}

Cryosphere fed irrigation networks are completely dependant on the timely availability of meltwater from
glaciers, snow and permafrost. With the accelerated decline of glaciers, these irrigation networks can no longer
deliver adequate water to sustain agricultural output and take advantage of the complete growing season. As a
consequence, some mountain villages have either been abandoned or lie on the brink of desertification.

In the past few decades, artificial ice reservoir (AIR) technologies have provided much needed relief to these
water-stressed communities. These strategies revolve around augmenting their glacial ice reservoirs with
man-made ones that provide supplementary irrigation during the spring. In the context of the observed present
and predicted global glacier shrinkage, the development of such water storage technologies is crucial to ensure
continued survival of mountain communities.

However, empirical data and studies focussing on AIRs are sparse. Depending on the relative influence of weather
conditions and fountain characteristics, AIRs typically show high variability in their ice volume dynamics.
Because small-scale processes, complex feedbacks and non-linearities govern their evolution, accessing the
response of AIRs to the location and fountain chosen is challenging and only feasible if backed up with
comprehensive field data. 

Modelling is a powerful tool that allows to identify the causes of AIR volume dynamics in space and time.
However, in-situ observations are required to tie these models to reality. The field measurement of ice volume
changes derived from drone flights on almost a dozen AIRs over many winters provide a unique data basis for
quantifying the feasibility and potential of this water storage technology worldwide.

The investigations of the meteorological forcing on the AIR surface were performed using a one dimensional mass
and energy balance model. The model has been shown to perform excellently when calibrated with field data.
Physical models provide unbiased insight into the physical processes and the drivers of temporal and spatial
changes.

It could be shown that the maximum volume of AIRs located in different regions may differ by an order of
magnitude. The differences can be attributed both to 

\section{Conclusions}

The main objective of this thesis was to improve our understanding about the response of AIRs to changes in
their construction location and fountain characteristics. With a special focus on icestupas, here defined as
vertical AIRs, a mass and energy balance model was developed and used as a tool to quantify the influence of
meteorological conditions and fountain characteristics between several measurement campaigns in India and
Switzerland. AIR radius, area and volume were recorded using DEMs produced from drone flights. The fountain
characteristics were calibrated from the observed radius and discharge rates. The model parameters were
calibrated from some volume observations. The rest of this DEM dataset were used to validate the modelled volume
evolution.

In a first experiment, the evolutional of AIRs in Indian Himalayas and the Swiss Alps are compared. The model
results show: 

\begin{itemize} 

\item[\tiny{$\blacksquare$}] Icestupa's maximum volume can vary by an order of magnitude due to meteorological
  conditions of their contruction location. 

\item[\tiny{$\blacksquare$}] Icestupa fountain systems spray five times more water than required.

\end{itemize}

In a second experiment, the evolutional of AIRs using different fountain scheduling strategies are compared. The
model results show: 

\begin{itemize} 

\item[\tiny{$\blacksquare$}] Weather-sensitive fountain systems increase AIR water-use efficiency 8 fold.

\item[\tiny{$\blacksquare$}] Weather-sensitive fountain systems enable maintenance-free AIR construction.


\end{itemize}

\section{Discussion}

\section{Recommendations}

\begin{itemize} 

\item[\tiny{$\blacksquare$}] Colder, drier and less cloudy construction locations form long-lasting AIRs with
  higher maximum ice volumes. 

\item[\tiny{$\blacksquare$}] Weather-sensitive fountain systems produce larger and efficient AIRs effortlessly. 

\end{itemize}

\section{Suggestions for future research}

\section{Final thoughts}
