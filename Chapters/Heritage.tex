\chapter{Heritage of ice reservoirs}

This chapter provides conclusions based on research findings from data collected on AIRs in Switzerland and India,
as well as discussion and recommendations for future research. This Chapter will review the purpose of the
study, research questions, literature review, and findings of the study. It will then present conclusions,
discussion of the conclusions, and recommendations for practice and for further research.

\section{Summary}

Cryosphere fed irrigation networks are completely dependant on the timely availability of meltwater from
glaciers, snow and permafrost. With the accelerated decline of glaciers, these irrigation networks can no longer
deliver adequate water to sustain agricultural output and take advantage of the complete growing season. As a
consequence, some mountain villages have either been abandoned or lie on the brink of desertification
\cite{grossmanHimalayanGlaciersMelt2015}.

In the past few decades, artificial ice reservoir (AIR) technologies have provided much needed relief to these
water-stressed communities. These strategies revolve around augmenting their glacial ice reservoirs with
man-made ones that provide supplementary irrigation during the spring. In the context of the observed present
and predicted global glacier shrinkage, the development of such water storage technologies is crucial to ensure
continued survival of mountain communities.

Published AIR observations and investigations date back to the mid-2000s \cite{tveitenGlacierGrowingLocal2007}.
The vast majority have been published in the 2010s, mostly using qualitative methods. Published quantifications
of their storage capacity differ widely \citep{baglaArtificialGlaciersHelp1998, norphelSnowWaterHarvesting2015,
nusserSociohydrologyArtificialGlaciers2019}.

We attempt to address this gap through a quantitative study on the ice stupa form of AIRs that takes into
account the influence of the location and fountain chosen. Because small-scale processes, complex feedbacks and
non-linearities govern their evolution, accessing the response of ice stupas is only feasible if backed up with
comprehensive field data. 

To address this data gap, we conducted measurement campaigns using drones, flowmeters and weather stations on almost a dozen
AIRs across two locations (India and Switzerland), over four winters (2019, 2020, 2021 and 2022) and using two
different construction methods (traditional and automated). The corresponding datasets were codenamed with the
prefix of the country, the suffix of the winter season and the construction method used (eg. Traditional CH21
AIR ). AIR radius, area and volume were recorded using DEMs produced from drone flights. The fountain
characteristics were calibrated from the observed radius and discharge rates. The model parameters were
calibrated from some volume observations. The rest of this DEM dataset were used to validate the modelled volume
evolution. This provided a unique data basis for quantifying the feasibility and potential of this water storage
technology worldwide.

Physical models provide unbiased insight into the physical processes and the drivers of temporal and spatial
changes. With a special focus on ice stupas, a mass and energy balance model was developed and used as a tool to
quantify the influence of meteorological conditions and fountain characteristics.  The model has been shown to
perform excellently when calibrated with field data.  It could be shown that the maximum volume of AIRs located
in the IN and CH regions differ by an order of magnitude. The differences can be attributed to the stronger
sublimation process due to the colder and drier weather conditions of the IN region. 

AIR maintenance requirement was reduced and their fountain freezing events were prevented by developing an
automated construction strategy. This weather-sensitive construction strategy made fountain operation efficient
and effortless through the use of an automation system that scheduled discharge rates based on the
recommendations of the AIR model. The automated construction strategy was successful in making AIRs using 87 \%
less water while being maintenance-free.

\section{Conclusions}

The main objective of this thesis was to improve our understanding about the response of AIRs to changes in
their construction location and fountain characteristics. In a first experiment, the evolution of AIRs in
Indian Himalayas and the Swiss Alps are compared. The model results show: 

\begin{enumerate} 

\item Icestupa's maximum volume can vary by an order of magnitude due to meteorological
  conditions of their construction location. 

\item Icestupa fountain systems spray five times more water than required.

\end{enumerate}

In a second experiment, the evolution of AIRs using different fountain scheduling strategies are compared. The
model results show: 

\begin{enumerate} 

\item Weather-sensitive fountain systems increase AIR water-use efficiency 8 fold.

\item Weather-sensitive fountain systems enable maintenance-free AIR construction.

\end{enumerate}

\section{Discussion}

\section{Recommendations}

\begin{itemize} 

\item[\tiny{$\blacksquare$}] Colder, drier and less cloudy construction locations form long-lasting AIRs with
  higher maximum ice volumes. 

\item[\tiny{$\blacksquare$}] Weather-sensitive fountain systems produce larger and efficient AIRs effortlessly. 

\end{itemize}

\section{Suggestions for future research}

\begin{itemize} 

\item[\tiny{$\blacksquare$}] Identification of favourable locations.

\item[\tiny{$\blacksquare$}] Cosistupa model development.


\end{itemize}

\section{Final thoughts}
